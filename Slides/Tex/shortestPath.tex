\documentclass[32pt,aspectratio=169]{beamer}

\usepackage[utf8]{inputenc} % Character encoding, choose your machine's default
                              % encoding; latin1 or utf8. If you want it read by
                              % others choose latin1.
\pdfinfo{
   /Author (Bashar Dudin)
   /Title  (Transportation problems)
   /Subject (Networks and Flows on Graphs)
}

\usepackage{./Style/My_Beamer} % This is my Beamer style.
\usepackage{./Style/Mystyle} % This is my own defined commands

%----------------------------------------------------------------------------------------
%   TITLE PAGE
%----------------------------------------------------------------------------------------

\author[BD]{Bashar Dudin}

\institute[]{EPITA}

\title{Networks and Flows on Graphs} %
\subtitle{Shortest Path Problems}

\begin{document}

\begin{frame}[plain]
\titlepage % Print the title page as the first slide
\end{frame}

\begin{frame}{What is it about?}
  \begin{halfshyblock}{Shortest path problem}
    Given a weighted directed graph a shortest path problem is about
    looking at paths having minimal lengths (weighted lenghts)
    \begin{itemize}
    \item[\textcolor<2->{lightgray}{\textbullet}] \textcolor<2->{lightgray}{between two given vertices (single-pair shortest path problem)}
    \item from a given source to any other vertex in the graph (single-source shortest path problem)
    \item between all pairs of vertices (all-pairs shortest path problem)
    \end{itemize}
  \end{halfshyblock}
  \pause[3]In order to be able to answer any of these problems the digraphs
  we're working with need not have any negative valued
  circuits (in case of single source, ones accessible from the source).
  \pause[4]
  \begin{rem}
    One could look for longest paths in a digraph. For graphs having no
    positive-valued circuits one can adapt the algorithms solving
    shortest path problems tweaking them a little.
  \end{rem}
\end{frame}

\begin{frame}[t]{Single-source shortest path problem : Ford's Algorithm}
  \small{
    \begin{algorithmic}[1]
      \Require $G$ a digraph having source $s$, $n$
      vertices and no negative-valued circuits accessible from $s$
      \Statex write $\nu$ the weight map from the set of arrows to integers
      \Ensure minimal lengths of paths from $s$
      \State number vertices arbitrarily $v_0$, $\ldots$, $v_{n-1}$ ensuring $s$ is the first
      \State initialize the vertex $v_0$ with $\lambda_0 =0$ and all others with $\lambda_i = +\infty$

      \Statex
      \For{$\_$ from $0$ to $n-1$}
      \ForAll{arrows $(v_\ell, v_j)$}
      \If{$\lambda_j > \lambda_\ell + \nu(v_\ell, v_j)$}
      \State $\lambda_j \gets \lambda_\ell + \nu(v_\ell, v_j)$
      \EndIf
      \EndFor
      \EndFor

      \State{\Return the list of $\lambda_i$s with the corresponding vertices}
    \end{algorithmic}
  }
  \tikzoverlay[text width=5.5cm] at (8.5cm, 4.8cm) {
    \begin{tcolorbox}[
      enhanced,
      parbox = false,
      colback=mLightBrown!10!white,
      colframe=mLightBrown,
      arc=0mm,
      ]
      \small{By adding an extra check to this algorithm one can also
        detect possible negative-valued circuits, without having to
        check it beforehand}
    \end{tcolorbox}%
  };
\end{frame}

\begin{frame}[t]{Ford's Algorithm : building a tree of shortest paths}
    \small{
      \begin{algorithmic}[1]
        \Require $G$ a digraph having source $s$, $n$ vertices and
        no negative-valued circuits accessible from $s$
        \Statex write $\nu$ the weight map from the set of arrows to integers
        \Ensure minimal lengths of paths from $s$ \alert{together with a tree $T$ of shortest paths}

       \State number vertices arbitrarily $v_0$, $\ldots$, $v_{n-1}$ ensuring $s$ is the first
       \State initialize the vertex $v_0$ with $\lambda_0 =0$ and all others with $\lambda_i = +\infty$
       \Statex \alert{$T$ is a tree having a single vertex $s$}
       \For{$\_$ from $0$ to $n-1$}
       \ForAll{arrows $(v_\ell, v_j)$}
       \If{$\lambda_j > \lambda_\ell + \nu(v_\ell, v_j)$}
       \State $\lambda_j \gets \lambda_\ell + \nu(v_\ell, v_j)$
       \State \alert{$T[j] \gets \ell$}
       \EndIf
       \EndFor
       \EndFor

       \State{\Return the list of $\lambda_i$s with the corresponding vertices \alert{and $T$}}
     \end{algorithmic}
     }
     \tikzoverlay[text width=5.5cm] at (8.5cm, 5.35cm) {
       \begin{tcolorbox}[
         enhanced,
         parbox = false,
         colback=mLightBrown!10!white,
         colframe=mLightBrown,
         arc=0mm,
         ]
         \small{The tree $T$ is built up by attaching to each vertex $k$ its
         predecessor $T[k]$, if there is any. To get a shortest path
         between $s$ and a vertex $v$ follow the unique path in $T$
         linking both.}
       \end{tcolorbox}%
     };
\end{frame}

\begin{frame}{Validity and complexity of Ford's algorithm}
  Let $V$ be the set of vertices of $G$ and $A$ its set of arrows.
  Line $2$ of Ford's algorithm takes $|V|$ operations. One goes $|V|$
  times through the loop at line $3$, for each iteration, one goes
  $|A|$ times through loop at line $4$. In total, we get a complexity
  for Ford's algorithm of
  $O\big(|V| + |V||A|\big) = O\big( |V||A|\big)$
  \begin{halfshyblock}{Validity}
    An elementary shortest path from $s$ to a vertex $v$ has length at
    most $|V|-1$. Given such an elementary path $a_0, \ldots, a_m$
    with $a_0=s$ and $a_m=v$ each sub-path from $s$ to any
    intermediate vertex is also a shortest path (otherwise we can make
    a shorter one from $s$ to $v$). Following Ford's algorithm along
    the previous path, one can show, by induction that $\lambda_m$ can
    only be equal to the (weighted) length of $a_0, \ldots, a_m$,
    i.e. the shortest path from $s$ to $v$.
  \end{halfshyblock}
\end{frame}


\begin{frame}{Single-source shortest path problem : Djikstra's algorithm}
  \small{
    \begin{algorithmic}[1]
      \Require $G$ a digraph having source $s$, $n$
      vertices and positive weight function $\nu$
      \Ensure minimal lengths of paths from $s$ to each vertex of $G$

      \State $V$ is the set of vertices of $G$ except for $s$
      \State initialize $s$ with $\lambda_s =0$, each of its
      successors $v$ with $\lambda_v = \nu(s, v)$
      \Statex and all other $v$ with $\lambda_v = +\infty$

      \While{$V \neq \emptyset$}
      \State find $v$ such that $\lambda_v$ is minimal among elements in $V$
      \ForAll{arrows $(v, v_j)$}
      \If{$\lambda_j > \lambda_v + \nu(v, v_j)$}
      \State $\lambda_j \gets \lambda_v + \nu(v, v_j)$
      \EndIf
      \EndFor
      \State $V \gets V\setminus \{v\}$
      \EndWhile

      \State{\Return the list of $\lambda$s with the corresponding vertices}
    \end{algorithmic}
  }
  \tikzoverlay[text width=5cm] at (9cm, 5.5cm) {
    \begin{tcolorbox}[
      enhanced,
      parbox = false,
      colback=mLightBrown!10!white,
      colframe=mLightBrown,
      arc=0mm,
      ]
      Building a tree of shortest paths can be adapted from what was
      done previously in the case of Ford's algorithm.
    \end{tcolorbox}%
  };
\end{frame}

\begin{frame}{Validity and Complexity of Djikstra's algorithm}
  A quick analysis of Djikstra's algorithm, as given here, shows it
  has complexity $O\big(|V|^2\big)$, where $V$ is the set of vertices
  of $G$. Indeed, the first loop is taken $|V|$ times, computing the
  minimum $\lambda_v$ and going through all your adjacent vertices
  during each loop, gives a final complexity of
  $O\Big(|V|\big(|V|+|A|\big)\Big)$ where $A$ is the set of arrows of
  $G$.
  \begin{rem}
    Implementation of Djikstra's algorithm can drop complexity to
    $O\big(|V|\lg|V| + |A|\big)$. This is not very used in practice
    since involved constants can be huge.
  \end{rem}
  \begin{halfshyblock}{Validity}
    Since all arrows are positively weighted, the minimal potential
    length $\lambda_v$ chosen at each step of line $4$ in Djikstra's
    algorithm would not have changed if we had run Ford's algorithm.
  \end{halfshyblock}
\end{frame}

\begin{frame}{All-pairs shortest path problem : Floyd-Warshall algorithm}
  We are looking for shortest paths between any two pairs of vertices
  in a given digraph $G$ with weighted arrows. \alert{Such paths exist
    as long as cycles are not negatively valued}.

  \vspace{.5\baselineskip}

  Number the vertices of $G$ from $1$ to $n$. Let $w_{ij}^k$
  be the minimal weight a path from $i$ to $j$ has, assuming
  intermediate vertices are in $\{1, \ldots, k\}$. If there is no such
  path $w_{ij} = +\infty$. If $P$ is a path joining $i$ to $j$ and having
  intermediate vertices in $\{1, \ldots, k \}$ then
  \begin{itemize}
  \item either $P$ does not go through $k$
  \item or $P$ goes only once through $k$ (only positive valued cycles);
    it is the concatenation of a path from $i$ to $k$ and one from $k$
    to $j$, only supported along $\{ 1, \ldots, k-1\}$.
  \end{itemize}
  This means that
  \begin{displaymath}
    w_{ij}^k = \min\Big( w_{ij}^{k-1}, w_{ik}^{k-1} + w_{kj}^{k-1} \Big)
  \end{displaymath}
\end{frame}

\begin{frame}{All-pairs shortest path problem : Floyd-Warshall algorithm}
    \small{
      \begin{algorithmic}[1]
       \Statex
       \Require $W$ the (generalized) adjacency matrix of a digraph $G$ having weighted arrows and only positve weighted cycles
       \Ensure minimal (weighted) lengths of paths between any two vertices of $G$
       \Statex
       \For{$k \in \{1, \ldots, n\}$}
       \For{$i \in \{1, \ldots, n\}$}
       \For{$j \in \{1, \ldots, n\}$}
       \State $W[i,j] \gets \min\Big(W[i,j], W[i,k] + W[k,j]\Big)$
       \EndFor
       \EndFor
       \EndFor
       \State \Return $W$
     \end{algorithmic}
     }
\end{frame}

\begin{frame}{Building up shortest paths using output of Floyd-Warshall algorithm}
  The output is a matrix $\Pi$ where entry at position $(i, j)$ gets a
  predecessor of $j$ along a shortest path from $i$ to $j$. To get a
  full shortest path from $i$ to $j$ read the matrix smartly.

  \vspace{.3\baselineskip}
    \small{
      \begin{algorithmic}[1]
        \Require Matrix $W$ of minimal lengths between any two pairs of vertices of $G$
        \Ensure A matrix $\Pi$ giving at entry $(i, j)$ a predecessor of
        $j$ along a shortest path from $i$ to $j$
        \For{$j \in \{1, \ldots, n\}$}
        \For{$i \in \{1, \ldots, n\}$}
        \ForAll{arrows $(v,j)$}
        \If{$W[i,j] \neq +\infty$ and $W[i,j] = W[i,v] + W[v,j]$}
        \State{$\Pi[i,j] \gets v$}
        \EndIf
        \EndFor
        \EndFor
        \EndFor
        \State{\Return{$\Pi$}}
      \end{algorithmic}
    }
\end{frame}

\begin{frame}
  \centering
  \Large \textbf{This is all we'll learn on this topic!}
\end{frame}

%%%%%%%%%%%%%%%%%%%%%%%%%%%%%%%%%%%%%%%%%%%%%%%%%%%%%%%%%%%%%%%

\end{document}

%%% Local Variables:
%%% mode: latex
%%% TeX-master: t
%%% End:
